%%%%%%%%%%%%%%%%%%%%%%%%%%%%%%%%%%%%%%%%%
% Journal Article
% LaTeX Template
% Version 1.3 (9/9/13)
%
% This template has been downloaded from:
% http://www.LaTeXTemplates.com
%
% Original author:
% Frits Wenneker (http://www.howtotex.com)
%
% License:
% CC BY-NC-SA 3.0 (http://creativecommons.org/licenses/by-nc-sa/3.0/)
%
%%%%%%%%%%%%%%%%%%%%%%%%%%%%%%%%%%%%%%%%%

%----------------------------------------------------------------------------------------
%	PACKAGES AND OTHER DOCUMENT CONFIGURATIONS
%----------------------------------------------------------------------------------------

\documentclass[twoside]{article}

\usepackage[sc]{mathpazo} % Use the Palatino font
\usepackage{amsmath}
\usepackage[T1]{fontenc} % Use 8-bit encoding that has 256 glyphs
\linespread{1.05} % Line spacing - Palatino needs more space between lines
\usepackage{microtype} % Slightly tweak font spacing for aesthetics

\usepackage[hmarginratio=1:1,top=32mm,columnsep=20pt]{geometry} % Document margins
\usepackage{multicol} % Used for the two-column layout of the document
\usepackage[hang, small,labelfont=bf,up,textfont=it,up]{caption} % Custom captions under/above floats in tables or figures
\usepackage{booktabs} % Horizontal rules in tables
\usepackage{float} % Required for tables and figures in the multi-column environment - they need to be placed in specific locations with the [H] (e.g. \begin{table}[H])
\usepackage{hyperref} % For hyperlinks in the PDF

\usepackage{lettrine} % The lettrine is the first enlarged letter at the beginning of the text
\usepackage{paralist} % Used for the compactitem environment which makes bullet points with less space between them

\usepackage{abstract} % Allows abstract customization
\renewcommand{\abstractnamefont}{\normalfont\bfseries} % Set the "Abstract" text to bold
\renewcommand{\abstracttextfont}{\normalfont\small\itshape} % Set the abstract itself to small italic text

\usepackage{titlesec} % Allows customization of titles
%\renewcommand\thesection{\Roman{section}} % Roman numerals for the sections
%\renewcommand\thesubsection{\Roman{subsection}} % Roman numerals for subsections
\titleformat*{\section}{\large\centering\bfseries} % Change the look of the section titles
\titleformat*{\subsection}{\large\bfseries} % Change the look of the section titles

\usepackage{fancyhdr} % Headers and footers
\pagestyle{fancy} % All pages have headers and footers
\fancyhead{} % Blank out the default header
\fancyfoot{} % Blank out the default footer
\fancyhead[C]{Running title $\bullet$ November 2012 $\bullet$ Vol. XXI, No. 1} % Custom header text
\fancyfoot[RO,LE]{\thepage} % Custom footer text
\usepackage[Table]{xcolor}
\usepackage{todonotes}
\usepackage{menukeys}
\usepackage[utf8]{inputenc}
\usepackage{pgfplots}
\usepackage{natbib}

\def\citeapos#1{\citeauthor{#1}'s (\citeyear{#1})}

%----------------------------------------------------------------------------------------
%	TITLE SECTION
%----------------------------------------------------------------------------------------

\title{\vspace{-15mm}\fontsize{24pt}{10pt}\selectfont\textbf{Automatic optimization of parameters for Ocular Artifact Correction in EEG}} % Article title

\author{
\large
\textsc{Benjamin Ahm, } % Your name
\textsc{Emil Riis Hansen, }
\textsc{Kristian Hauge Jensen, }\\
\textsc{Morten Korsholm Terndrup}\\[2mm]\thanks{A thank you or further information}
\normalsize Aalborg University \\ % Your institution
\normalsize \href{mailto:mternd13@student.aau.dk}{mternd13@student.aau.dk} % Your email address
\vspace{-5mm}
}
\date{}

%----------------------------------------------------------------------------------------

\begin{document}

\maketitle % Insert title

\thispagestyle{fancy} % All pages have headers and footers

%----------------------------------------------------------------------------------------
%	ABSTRACT
%----------------------------------------------------------------------------------------

\begin{abstract}

\noindent Here is the abstract.

\end{abstract}

%----------------------------------------------------------------------------------------
%	ARTICLE CONTENTS
%----------------------------------------------------------------------------------------

\begin{multicols}{2} % Two-column layout throughout the main article text


\section{Introduction}

\lettrine[nindent=0em,lines=3]{L} orem ipsum dolor sit amet, consectetur adipiscing elit.



\subsection{Related Work}



%------------------------------------------------

\section{Optimization of parameters for Ocular Artifact Correction}
Ocular artifacts such as eye movements or blinking are often present in EEG data, and is the cause for significant decrease in classification accuracy. The reason for this, is that the amplitude of a signal changes when eye movements happen and can introduce uncertainty about the events we are interested in classifying, such as motor imagery. In order to reduce the impact of these ocular artifacts, we use the OACL technique proposed by \citet{li2015ocular} and generalize it to handle multi-class EEG data.
\subsection{Ocular Artifact Correction}
General intro to oacl. What is the approach.

\subsubsection{Artifact Detection}
Explain how we detect artifacts and get artifact signals.

\subsubsection{Artifact Removal}
Explain how we use the artifact signals to remove artifacts from the eeg signal.
\section{Filter Bank Common Spatial Patterns}
\todo{You should describe a bit more about Filter Bank and Multi Class CSP, at least 2 paragraphs for each technique.}
We have chosen to use a Filter bank multi class common spatial patterns (FBMCCSP) algorithm, for the purpose of extracting features from EEG data. The CSP algorithm has been used in many EEG classification studies within recent years, as in \cite{ang2012filter}. In all studies,\todo{source?} CSP were considered a step of improvement to the classification of EEG classes. Based on these results, we chose to incorporate CSP as part of the classification pipeline. CSP assumes the user knows which frequency ranges contains important features for different imagery classes. Since one rarely knows of these ranges, we have implemented the filter bank algorithm, as a step before CSP, as this automatically chooses frequency ranges from EEG data. Filter bank is also a well studied algorithm, which is often used in conjunction with CSP, as in \cite{ang2008filter}. CSP can in its original form, only be used in binary classification problems, but the extension explained in this paper, multi class CSP (MCCSP), does not have that restriction. Since MCCSP makes use of the original binary CSP, we first explain how CSP works, and then extend it with multi class.

\begin {figure*}%[!hbtp]
\centering
\begin{adjustbox}{width=\textwidth}
\begin{tikzpicture}

% Variables
\pgfmathsetmacro{\bs}{0.5};
\pgfmathsetmacro{\boxl}{2};
\pgfmathsetmacro{\boxh}{1};
\pgfmathsetmacro{\ll}{1};
\coordinate (blength) at (0.5, 0);
\coordinate (linel) at (1, 0);
\coordinate (bh) at (0, 1);
\coordinate (bl) at (2, 0);
\newcommand*{\fblist}{-3, 0, 3}
\newcommand*{\csplist}{-1, 0, 1}
\newcommand*{\ovrlist}{{af},{bo},{co}}
	

% Coordinate for start circle
\coordinate (trains) at (0, 0);

% Coordinate for Bayesian Optimization
\coordinate (bos) at ($(trains) + (blength) + 3/2*(linel)$);

% Coordinate for cross validation
\coordinate (crosss) at ($(bos) + (blength) + 3/2*(linel) + 1/2*(bl)$);

% Coordinate for Ocular Articaft Correction
\coordinate (oacls) at ($(crosss) + (bl) + (linel)$);

% Coordinate for Filter Bank
\coordinate (filters) at ($(oacls) + (bl) + (linel)$);

% Coordinate for filterbank nodes
\coordinate (filterbanks) at ($(filters) + 1/2*(bl) + 1.5*(linel) + 1/2*(blength)$);

% Coordinate for csp ovr nodes
\coordinate (cspovrs) at ($(filterbanks) + 1/2*(blength) + (linel)$);

% Coordinate for Random Forest Learner nodes
\coordinate (randomforestlearner) at ($(cspovrs) + 1/2*(blength) + 3.5*(linel)$);

% Coordinate for result node
\coordinate (results) at ($(randomforestlearner) + 1/2*(bl) + 1/2*(blength) + (linel)$);

% Coordinate for mean results
\coordinate (meanresults) at ($(results) + (blength) + (linel)$);

% Cooordinates for start and end of step box
\coordinate (boxceil) at (0, 6);
\coordinate (boxfloor) at (0, -6);
\coordinate (startbox) at ($(trains) + (-1, 0)$);
\coordinate (endbox) at ($(meanresults) + (1, 0)$);


% Draw training data circle
\node [draw, label={Train Data}, circle, name=traincircle, minimum size = \bs] at (trains) {};

% Draw Bayesian Optimization Box
\node (bo) at (bos) [draw,thick,minimum width=\boxl cm,minimum height=\boxh cm] {BO};
\draw [->] (traincircle) -- (bo);

% Draw Cross validation box
\node (crossvalidation) at (crosss) [draw,thick,minimum width=\boxl cm,minimum height=\boxh cm] {CV};
\draw [->] (bo) -- (crossvalidation);

% Draw oacl box
\node (oacl) at (oacls) [draw,thick,minimum width=\boxl cm,minimum height=\boxh cm] {OAC};
\draw [->] (crossvalidation) -- (oacl);

% Draw Filter Bank
\node (filterbank) at (filters) [draw,thick,minimum width=\boxl cm,minimum height=\boxh cm] {FB};
\draw [->] (oacl) -- (filterbank);

% Draw filterbank nodes
\foreach \x in \fblist{
	\node [draw, circle, name=filterbanknode\x, minimum size = \bs] at ($(filterbanks) + (0, \x)$) {};
}

\draw [->] (filterbank) -- node[above] {[4, 8]} (filterbanknode-3);
\draw [->] (filterbank) -- node[above] {[8, 12]} (filterbanknode0);
\draw [->] (filterbank) -- node[above] {[12, 16]} (filterbanknode3);

% Draw Filter Bank Nodes and CSP OVR nodes
\foreach \x in \fblist
	\foreach \y in \csplist{
		\node [draw, circle, name=cspovrnode\x\y, minimum size = \bs] at ($(filterbanknode\x) + (0, \y) + 2*(linel)$) {};
		\draw [->] (filterbanknode\x) -- (cspovrnode\x\y);
}

% Draw names of csp
\foreach \x in \fblist{
	\draw [->] (filterbanknode\x) -- node[above] {1-23} (cspovrnode\x-1);
	\draw [->] (filterbanknode\x) -- node[above] {2-13} (cspovrnode\x0);
	\draw [->] (filterbanknode\x) -- node[above] {3-12} (cspovrnode\x1);
}


% Draw filter bands
\noindent\foreach [count=\i] \x in \fblist{
	\draw [->] (filterbank) -- node[above] {} (filterbanknode\x);
}

% Draw random forest classifier node
\node (randomforestnode) at (randomforestlearner) [draw,thick,minimum width=\boxl cm,minimum height=\boxh cm] {RFC};

% Draw CSP OVR arrows to classifier node
\foreach \x in \fblist
\foreach \y in \csplist{
	\draw [->] (cspovrnode\x\y) -- (randomforestnode);
}

% Draw result node
\node [draw, label={Fold result}, circle, name=result, minimum size = \bs] at (results) {};
\draw [->] (randomforestnode) -- (result);

% Draw mean result node
\node [draw, label={south:Mean Result}, circle, name=meanresult, minimum size = \bs] at ($(meanresults) + (0, -1)$) {};
\draw [->] (result) -- (meanresult);

% Draw curved arrows
\draw [->] (result) to[out=270, in=270, distance=165] (crossvalidation);
\draw [->] (meanresult) to[out=270, in=270, distance=180] (bo);

% Draw box around image and horizontal lines
\draw ($(startbox) + (boxfloor)$) -- ($(endbox) + (boxfloor)$) -- ($(endbox) + (boxceil)$) -- ($(startbox) + (boxceil)$) -- cycle;

\draw[loosely dotted] ($(trains) + (boxfloor) + 1/2*(linel)$) -- ($(trains) + (boxceil) + 1/2*(linel)$);
\node[draw] at ($(trains) + (boxceil) - (0, 1)$) {1};

\draw[loosely dotted] ($(bos) + (boxfloor) + 3/2*(linel)$) -- ($(bos) + (boxceil) + 3/2*(linel)$);
\node[draw] at ($(bos) + (boxceil) - (0, 1)$) {2};

\draw[loosely dotted] ($(crosss) + (boxfloor) + 3/2*(linel)$) -- ($(crosss) + (boxceil) + 3/2*(linel)$);
\node[draw] at ($(crosss) + (boxceil) - (0, 1)$) {3};

\draw[loosely dotted] ($(oacls) + (boxfloor) + 3/2*(linel)$) -- ($(oacls) + (boxceil) + 3/2*(linel)$);
\node[draw] at ($(oacls) + (boxceil) - (0, 1)$) {4};

\draw[loosely dotted] ($(filters) + (boxfloor) + 3/2*(linel)$) -- ($(filters) + (boxceil) + 3/2*(linel)$);
\node[draw] at ($(filters) + (boxceil) - (0, 1)$) {5};

\draw[loosely dotted] ($(filters) + (boxfloor) + 15/4*(linel)$) -- ($(filters) + (boxceil) + 15/4*(linel)$);
\node[draw] at ($(filters) + (boxceil) - (0, 1) +11/4*(linel)$) {6};

\draw[loosely dotted] ($(filterbanks) + (boxfloor) + 3*(linel)$) -- ($(filterbanks) + (boxceil) + 3*(linel)$);
\node[draw] at ($(filterbanks) + (boxceil) - (0, 1) +2*(linel)$) {7};

\draw[loosely dotted] ($(randomforestlearner) + (boxfloor) + 3/2*(linel)$) -- ($(randomforestlearner) + (boxceil) + 3/2*(linel)$);
\node[draw] at ($(randomforestlearner) + (boxceil) - (0, 1)$) {8};

\draw[loosely dotted] ($(result) + (boxfloor) + 2/3*(linel)$) -- ($(result) + (boxceil) + 2/3*(linel)$);
\node[draw] at ($(result) + (boxceil) - (0, 1)$) {9};

\node[draw] at ($(meanresults) + (boxceil) - (0, 1)$) {10};

\end{tikzpicture}
\end{adjustbox}
\caption{Overview of program pipeline}
\label{fig:ProgramPipeline}
\end{figure*}

\subsection{Common Spatial Patterns}\label{sec:csp}
CSP finds spatial filters, which when applied to signals, gives the maximal mutual information between these, with respect to signal variance. The method assumes there are classification information hidden within the variance between signals. Assuming we are classifying on motor imagery for different body parts, as we are in the training and evaluation data, this assumption can be justified \citep{blankertz2008optimizing}.
Formally CSP combines data trials with the same imagery task. Let $\pmb{A}$ and $\pmb{B}$ be matrices of combined trials for imagery task 1 and 2 respectively,

\begin{equation}
\label{eq:csp_data}
\pmb{A}, \mathbf{B} \in \mathbb{R}^{n*m}
\end{equation}
where $n$ and $m$ are the number of signals and samples respectively. CSP now calculates the covariance matrices for $\pmb{A}$ and $\pmb{B}$,

\begin{equation}
\label{eq:covariance_matrice}
\pmb{A_{cov}} = \frac{(\pmb{A} \cdot \overline{\pmb{A}})^\mathsf{T}  \cdot (\pmb{A} \cdot \overline{\pmb{A}})}{m - 1}
\end{equation}
where $m$ is the number of samples in $\pmb{A}$, and elements of $\overline{\pmb{A}}$ is defined as,

\begin{equation}
\label{eq:a_bar}
\pmb{\overline{A}_{ij}} = \frac{\pmb{A_{i,1}} + \pmb{A_{i,2}} + ... + \pmb{A_{i,m}}}{m}
\end{equation}

By applying simultaneous diagonalization between $\pmb{A_{cov}}$ and $\pmb{B_{cov}}$, we form the eigenvectors $\pmb{P}$, which will be the spatial filters for maximizing variance between class 1 and 2. $\pmb{P}$ is found when both of the following diagonalizations hold, 

\begin{equation}
\label{eq:diagonalization_A}
\pmb{P} \cdot \pmb{A_{cov}} \cdot \pmb{P} = \pmb{D}, \quad \pmb{P}, \pmb{D}, \pmb{A_{cov}} \in \mathbb{R}^{n*n}
\end{equation}

\begin{equation}
\label{eq:diagonalization_B}
\pmb{P} \cdot \pmb{B_{cov}} \cdot \pmb{P} = \pmb{I}, \quad \pmb{P}, \pmb{I}, \pmb{B_{cov}} \in \mathbb{R}^{n*n}
\end{equation}

The spatial filter will now correspond to the first row of $\pmb{P}$, such that $\pmb{\vec{w}} = \pmb{P}^\mathsf{T}_{1}$ 

$\pmb{P}$ can now be used as a linear transformation which when applied to EEG signals, maps these into a new space, where signal features are more discriminative. The drawback of CSP is that it does not work well if the frequency bands are not adjusted to fit each subject \citep{novi2007sub}. Furthermore, CSP only works for two classes, whereas many real world applications require a greater number of classes.

\subsection{Filter bank}
Finding good frequency bands for each subject can be quite time-consuming when done manually. Fortunately, the process can be automated by creating a filter bank (FB) that splits a signal into components, each of which contains a frequency sub-band. This can be seen in \cref{fig:ProgramPipeline} as step 5.

The sub-bands are chosen within the frequency range of 4 to 40, which should assure all relevant data is taken into account \citep{pfurtscheller1999event}. They are chosen with an interval of $n \in \{3,..,8\}$. For an interval of $n = 3$ we would create a set $F$ of filtersss, $F \in \{[4, 7], [7, 10],...,[37, 40]\}$. Every filter will be used in the creation of CSPs, which will form the basis of feature extraction.

\subsection{Multi-class CSP}
We introduce MCCSP by the one versus rest (OVR) method. The method constructs one CSP per class, by choosing a class, and treating every other class as being the same. This way we get one CSP per imagery class, each constructed to create the maximum variance to all other classes. The method is depicted in \Cref{fig:ProgramPipeline} as step 6. The example figure shows a pipeline with 3 filter bands, and three classes, from which 9 CSP are created. EEG features can now be found, by applying the spatial filters from CSP, to EEG trials. Our approach is to apply each CSP, to every trial, and combine all of their respective feature vectors. As an example, say we have $m$ CSP, after applying OVR over all classes, for all bands. If now we apply one of the $m$ CSP to a single trial, we get a feature vector with $n$ components. By applying all CSP to the same trial, and combining their feature vectors, we get a trial with $n * m$ features. We apply this method on all trials in the dataset, from where $n * m$ features are found, for every trial. These can now be used to train a random forest classifier.     








\section{Bayesian Optimization}\label{sec:bayesian-optimization}
Bayesian optimization is a method for finding the extrema of expensive functions. In machine learning, we can see classification algorithms as functions to be optimized over their hyperparameters with respect to model accuracy, and thus Bayesian optimization can be used to find the combination of hyperparameters that yields the highest classification accuracy. \citet{snoek2012practical} show that Bayesian optimization can be applied to existing problems and in some cases outperform even experts at tuning machine learning algorithms. Other benefits of Bayesian optimization is the fairness when evaluating algorithms against each other as well as configuration of algorithms being more reproducible.
\todo{insert the simple algorithm for bayesian optimization}
BO sees the objective function as a black box. Generally, it tries to fit a \emph{Gaussian process} (GP) to the unknown objective function by requesting results at various parameter settings, and eliminating GP samples that do not fit the solution. The remaining GP samples are then used to form a posterior distribution over the objective function. An acquisition function can now be formed by probing the surrogate function. The next parameter setting for probing the objective function will now be the maximum expected result gain from the acquisition function.
\begin{algorithm}
	\For{$x\in X$}{
		$NbSuccInS(x) \longleftarrow 0$\;
		$NbPredInMin(x) \longleftarrow 0$\;
		$NbPredNotInMin(x) \longleftarrow |ImPred(x)|$\;
	}
\end{algorithm}
\subsection{Gaussian Processes}
A random variable is a probability distribution over an event. One example is the random variable $CoinFlip = (0.5, 0.5)$ with a 50\% chance of either heads or tails. Such probability distributions can be Gaussian, e.g. the outcomes of the event tend to cluster around some mean value and distribute evenly on either side of the mean. Generalizing the notion of Gaussians, two random variables can also be jointly Gaussian or multivariate Gaussian in their covariances. A Gaussian process over a set S, is a set of random variables $GP = (Z_t | t \in S)$ such that all linear combinations of $Z_t$ are multivariate Gaussian. This means that we can interpret a Gaussian process as a probability distribution over (Gaussian) functions e.g. given a $t \in S$ we can get the function describing the probability distribution of variable $Z_t$. Since a Gaussian is defined by the mean value $\mu$ and variance $\sigma^2$, we can consider a Gaussian process as a function $GP : X \rightarrow \mathbb{R} \times \mathbb{R}$ where X is the set of combinations of hyperparameters:

\begin{equation}\label{gaussian-process}
GP(x) = (m(x), k(x, x'))
\end{equation}
where $m : X \rightarrow \mathbb{R}$ is the mean function, and $k : X \times X \rightarrow R^n$ is the \emph{covariance function} (also called the kernel function) of the Gaussian process.

An example of a Gaussian process over one hyperparameter x is seen in figure x. 

\subsection{Kernel function}
The kernel is a function for the GP, which determines the smoothness of samples drawn from it. The choice of kernel is crucial in the process of finding the right surrogate function, to fit the objective function. Many kernels have been proposed, where one of the most used is the squared exponential function \citet{brochu2010tutorial}. The function however tends to smooth the surrogate function to much, to be applicable in real world problems. The kernel function used in this method will therefor be Matérn$\frac{5}{2}$, as proposed by \citet{snoek2012practical}.

\subsection{Acquisition function}
When Bayesian optimization builds its model of the objective function, it iteratively choses inputs to sample outputs from the objective function. The choice of which inputs to sample from is determined by the \emph{acquisition function} (AF), which determines utility of sampling from a given input. Different acquisition functions yield different measures to make this decision. One measure is the \emph{Probability of Improvement} (PI) which given a candidate input computes the probability of improving the current best result. \todo{explain with figure} Another approach is to consider the \emph{expected improvement} (EI) which not only takes into account the probability of improvement, but also the uncertainty e.g. the variance of the Gaussian distribution of the surrogate at the given input.
EI balances the trade-off between exploitation and exploring, and is therefor a well used acquisition function \citet{brochu2010tutorial}. EI(x) is the function which gives the expected improvement of choosing parameters x, and is defines as:

\begin{equation}
\label{eq:expected-improvement}
EI(x) =
\begin{cases}
   (K + L & \text{ if } \sigma(x) > 0\\
   0 	  & \text{ if } \sigma(x) = 0
\end{cases}
\end{equation}
where $K = (\mu(x) - f(x^+))\Phi(Z)$ \\and $L = \sigma(x)\phi(Z)$ .
$\mu$ and $\sigma$ are the mean and variance of the posterior distribution of the surrogate function, respectively. $Z$ is defined as:

\begin{equation}
\label{eq:expect-z}
Z =
\begin{cases}
\frac{\mu(x) - f(x^+)}{\sigma(x)} & \text{ if } \sigma(x) > 0\\
0 								  & \text{ if } \sigma(x) = 0
\end{cases}
\end{equation}
$\phi$ and $\Phi$ denote the \emph{probability density function} (PDF) and \emph{cumulative distribution function} (CDF) of the standard normal distribution.
The function for EI can then be used to probe the surrogate function with different parameter settings x, from where a new candidate point can be found to probe the objective function. The new candidate will be the parameter setting with the highest expected improvement.  


%- Combining posterior and refit the GP

\section{Random Forest Classification}

With the feature vector $\mathbf{X}$ extracted by the multi-class CSP algorithm as the training set, we train a classifier for multi-class motor imagery.

Several algorithms have seen popularity for classifying EEG data. The survey by \citet{chan2015systematic} on the performance of ensemble methods in EEG context argues that Random Forests more accurately classifies EEG data than other well-known methods such as k nearest neighbors and Support Vector Machines. \citet{sun2007experimental} also surveys the effectiveness of ensemble methods, but argues that performance is subject to the choice of base classifier as weak learners. To evaluate the results we train a Random Forrest classifier. We use the implementation from the Scikit-Learn library for Python [\citet{scikit-learn}].   

The Random Forrest learning algorithm works by splitting the training set $T$ into $n$ subsets $\{t_1,…,t_n \quad | \quad t_i \subset S\}$ and trains $n$ decision trees for each subset $t_i$. The splitting is done randomly by drawing a \emph{bootstrap sample} with replacement e.g. sets are constructed by uniformly chosing samples, with the possibility that one sample is drawn more than once. When classifying new samples, the Random Forrest classifies on each of its decision trees and returns the mode result. Intuitively, the weak learners 'vote' on the result.

The decision trees are constructed by randomly splitting the training subset $t_i$ on the features to obtain the training subset $t’_i \subset t_i$ containing the feature values of the randomly chosen features, in the given subset. The decision tree is then constructed according to the C4.5 algorithm which constructs a node on the split with the highest information gain, and finally pruned for features which do not provide any improvements in model accuracy.

The hyperparameters in a Random Forest classifier is the number of weak learners used to build the classifier. Generally, the more weak learners used the better the model accuracy and for this reason we use one tree for each feature in our feature space given by the CSP algorithm. 

%------------------------------------------------

\section{Experimental Results}
In order to evaluate our method, we use the BCI Competition IV dataset 2a \citep{brunner2008bci}, which contains 4-class motor imagery EEG data from 9 subjects. The dataset consist of labeled training and test sets. Each subject participated in two sessions of 6 runs on different days. A run consist of 48 labeled trials, divided evenly between the 4 classes. Each trial measured the brain signals of a subject on 22 EEG channels and 3 EOG channels. We discarded the EOG channels since we are interested in correcting artifacts without any reference signals.

We set up two pipelines to be compared. The first consists of OACL, followed by FBCSP and Random Forest. The second is without the OACL step, i.e., just FBCSP and Random Forest. We perform 6-fold cross-validation on the training data using five runs for training and one for validation. We run 200 iterations of Bayesian Optimization of select hyperparameters for each pipeline. The best hyperparameters for each pipeline is then used for the final evaluation. We repeat this for each subject. \Cref{fig:results} shows the obtained accuracies and KAPPA scores for each subject.

\begin{table}[H]
	\centering
	\caption{Accuracy and KAPPA score for each subject.}
	\label{fig:results}
	\begin{tabular}{@{}ccc|cc@{}}
		\toprule
		\textbf{S}             & \multicolumn{2}{c|}{\textbf{W. BO OACML}} & \multicolumn{2}{c}{\textbf{W/O BO OACML}} \\ \midrule
		\multicolumn{1}{c|}{}  & Acc                   & KAPPA             & Acc                   & KAPPA             \\ \midrule
		\multicolumn{1}{c|}{1} &                       &                   &                       &                   \\
		\multicolumn{1}{c|}{2} &                       &                   &                       &                   \\
		\multicolumn{1}{c|}{3} &                       &                   &                       &                   \\
		\multicolumn{1}{c|}{4} &                       &                   &                       &                   \\
		\multicolumn{1}{c|}{5} & \textbf{}             &                   & \textbf{}             &                   \\
		\multicolumn{1}{c|}{6} &                       &                   &                       &                   \\
		\multicolumn{1}{c|}{7} &                       &                   &                       &                   \\
		\multicolumn{1}{c|}{8} &                       &                   &                       &                   \\
		\multicolumn{1}{c|}{9} &                       &                   &                       &                   \\ \bottomrule
	\end{tabular}
\end{table}

To determine the significance of these results we use the Wilcoxon signed-rank test. \Cref{fig:wilcoxon} shows the results of the test.

\begin{table}[H]
	\centering
	\caption{Wilcoxon signed-rank test}
	\label{fig:wilcoxon}
	\begin{tabular}{@{}l|llll@{}}
		\toprule
		S & No OACL & OACL & Diff & Rank \\ \midrule
		&               &                 &      &      \\
		&               &                 &      &      \\
		&               &                 &      &      \\
		&               &                 &      &      \\
		&               &                 &      &      \\
		&               &                 &      &      \\
		&               &                 &      &      \\
		&               &                 &      &      \\
		&               &                 &      &      \\ \bottomrule
	\end{tabular}
\end{table}

\subsection{Discussion}
Here we discuss the results given in section 3, and talk more about what the results imply/how it could be improved.
Remember to discuss the issues of complexity of experiment vs. how much effort we put into selecting the "best" candidate in Bayesian Optimization

%------------------------------------------------

\section{Conclusion}
\todo{to be done}

%----------------------------------------------------------------------------------------
%	REFERENCE LIST
%----------------------------------------------------------------------------------------
\bibliographystyle{te}
\bibliography{references}

%----------------------------------------------------------------------------------------

\end{multicols}

\end{document}
