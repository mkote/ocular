\section{Experimental Results}
In order to evaluate the Bayesian Optimized Ocular Artifact Correction, we train and test the technique on the BCI Competition dataset IV 2a of 4-class motor imagery EEG data. The dataset consist of labeled training and test sets. The training set consists of EEG measurements for 9 subjects. The 9 subjects each participated in 2 sessions of 6 runs. A run consist of 48 labeled trials, divided between each of the 4 classes, which yields 12 trials for each label. Each trial measured the brain signals of a subject on 22 EEG channels and 3 EOG channels. We discarded the EOG channels since we are interested in correcting artifacts without any secondary signal sources.

To see the effects of OACL, we the following experiments were setup. Two pipelines were set up where pipeline 1 performs Bayesian Optimization on OACL with FBCSP and Random Forest Classification and pipeline 2 is identical except without the OACL method applied.

To investigate the generality properties, we performed 6-fold cross-validation on the dataset where each fold contains the data of a single run. We performed the experiments on individual subjects to obtain the respective classification accuracies for each pipeline.
\begin{table}[H]
	\centering
	\caption{Accuracies and KAPPA score on the set S of subjects.}
	\label{fig:results}
	\begin{tabular}{@{}ccc|cc@{}}
		\toprule
		\textbf{S}             & \multicolumn{2}{c|}{\textbf{W. BO OACML}} & \multicolumn{2}{c}{\textbf{W/O BO OACML}} \\ \midrule
		\multicolumn{1}{c|}{}  & Acc                   & KAPPA             & Acc                   & KAPPA             \\ \midrule
		\multicolumn{1}{c|}{1} &                       &                   &                       &                   \\
		\multicolumn{1}{c|}{2} &                       &                   &                       &                   \\
		\multicolumn{1}{c|}{3} &                       &                   &                       &                   \\
		\multicolumn{1}{c|}{4} &                       &                   &                       &                   \\
		\multicolumn{1}{c|}{5} & \textbf{}             &                   & \textbf{}             &                   \\
		\multicolumn{1}{c|}{6} &                       &                   &                       &                   \\
		\multicolumn{1}{c|}{7} &                       &                   &                       &                   \\
		\multicolumn{1}{c|}{8} &                       &                   &                       &                   \\
		\multicolumn{1}{c|}{9} &                       &                   &                       &                   \\ \bottomrule
	\end{tabular}
\end{table}
\Cref{fig:results} shows the obtained accuracies and KAPPA scores on each individual subject after cross-validation on the two pipelines.
\begin{table}[H]
	\centering
	\caption{Rank Sum}
	\label{fig:wilcoxon}
	\begin{tabular}{@{}l|llll@{}}
		\toprule
		S & No OACL & OACL & Diff & Rank \\ \midrule
		&               &                 &      &      \\
		&               &                 &      &      \\
		&               &                 &      &      \\
		&               &                 &      &      \\
		&               &                 &      &      \\
		&               &                 &      &      \\
		&               &                 &      &      \\
		&               &                 &      &      \\
		&               &                 &      &      \\ \bottomrule
	\end{tabular}
\end{table}
\Cref{fig:wilcoxon} shows the ranks in the Wilcoxon test.

\subsection{Discussion}
Here we discuss the results given in section 3, and talk more about what the results imply/how it could be improved.
Remember to discuss the issues of complexity of experiment vs. how much effort we put into selecting the "best" candidate in Bayesian Optimization