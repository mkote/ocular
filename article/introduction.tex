\section{Introduction}
The field of Brain-Computer Interfaces (BCI) has in the recent years been under active research, especially with the popularity of machine learning techniques. The reason for the interest, is the many useful application of a well-working BCI, such as replacing lost motor function in disabled people, helping with analysis in brain imaging to diagnose brain conditions or novel applications in computer games. 

The general idea of a BCI is to measure brain activity represented by electroencephalogram (EEG) signals, by putting sensors on the scalp, which can measure the electric impulses. However, the EEG data is noisy at best, and this problem can severely affect the results of classification algorithms. Therefore, signal processing is an important step in any given BCI.

All in all, this leaves us with several steps in which several techniques may be applied to obtain the corrected EEG signal. Each technique applied may require several parameters to be tuned for obtaining the optimal results. Users of the BCI or medical professionals are usually knowledgeable about tuning some of the parameters but not all, hence it requires either an expert to help determine them or extensive training.
Another, more useful approach would be to automatically infer the hyper-parameters from the training data. Recent work about algorithmically optimizing machine learning parameters has seen popularity by using Bayesian Optimization (insert reference), in which the learning process is seen as a black-box function for which parameters can be optimized.

\subsection{Related Work}
Much research effort has been put into developing or applying techniques for noise/artifact correction in EEG signals. Well-known methods, such as \textit{Principal Component Analysis}, \textit{Independent Component Analysis} or \textit{Discrete Wavelet Transform}, in the signal processing world has had mixed results. Most of these techniques considers how to extract the information from the signals, instead of reducing the noise in the data.
Other approaches such as (source) considers removing noise from a correction perspective. [ECG source] uses EOG signals measured by sensors on the subjects eyes, and estimates a propagation factor to determine the amount of the EOG signal to remove from the EEG signal, thus using the EOG signal as an artifact signal. Similarly, [OACL guys] have had positive results in estimating a pseudo-EOG signal for binary class EEG data, making it possible to obtain an artifact signal without using any secondary measurements.
% Someone proposed to make several passes over eeg to remove a single type of artifact
