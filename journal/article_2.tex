%%%%%%%%%%%%%%%%%%%%%%%%%%%%%%%%%%%%%%%%%
% Journal Article
% LaTeX Template
% Version 1.3 (9/9/13)
%
% This template has been downloaded from:
% http://www.LaTeXTemplates.com
%
% Original author:
% Frits Wenneker (http://www.howtotex.com)
%
% License:
% CC BY-NC-SA 3.0 (http://creativecommons.org/licenses/by-nc-sa/3.0/)
%
%%%%%%%%%%%%%%%%%%%%%%%%%%%%%%%%%%%%%%%%%

%----------------------------------------------------------------------------------------
%	PACKAGES AND OTHER DOCUMENT CONFIGURATIONS
%----------------------------------------------------------------------------------------

\documentclass[twoside]{article}

\usepackage{lipsum} % Package to generate dummy text throughout this template

\usepackage[sc]{mathpazo} % Use the Palatino font
\usepackage[T1]{fontenc} % Use 8-bit encoding that has 256 glyphs
\linespread{1.05} % Line spacing - Palatino needs more space between lines
\usepackage{microtype} % Slightly tweak font spacing for aesthetics

\usepackage[hmarginratio=1:1,top=32mm,columnsep=20pt]{geometry} % Document margins
\usepackage{multicol} % Used for the two-column layout of the document
\usepackage[hang, small,labelfont=bf,up,textfont=it,up]{caption} % Custom captions under/above floats in tables or figures
\usepackage{booktabs} % Horizontal rules in tables
\usepackage{float} % Required for tables and figures in the multi-column environment - they need to be placed in specific locations with the [H] (e.g. \begin{table}[H])
\usepackage{hyperref} % For hyperlinks in the PDF

\usepackage{lettrine} % The lettrine is the first enlarged letter at the beginning of the text
\usepackage{paralist} % Used for the compactitem environment which makes bullet points with less space between them

\usepackage{abstract} % Allows abstract customization
\renewcommand{\abstractnamefont}{\normalfont\bfseries} % Set the "Abstract" text to bold
\renewcommand{\abstracttextfont}{\normalfont\small\itshape} % Set the abstract itself to small italic text

\usepackage{titlesec} % Allows customization of titles
%\renewcommand\thesection{\Roman{section}} % Roman numerals for the sections
%\renewcommand\thesubsection{\Roman{subsection}} % Roman numerals for subsections
\titleformat*{\section}{\large\centering\bfseries} % Change the look of the section titles
\titleformat*{\subsection}{\large\bfseries} % Change the look of the section titles

\usepackage{fancyhdr} % Headers and footers
\pagestyle{fancy} % All pages have headers and footers
\fancyhead{} % Blank out the default header
\fancyfoot{} % Blank out the default footer
\fancyhead[C]{Running title $\bullet$ November 2012 $\bullet$ Vol. XXI, No. 1} % Custom header text
\fancyfoot[RO,LE]{\thepage} % Custom footer text

%----------------------------------------------------------------------------------------
%	TITLE SECTION
%----------------------------------------------------------------------------------------

\title{\vspace{-15mm}\fontsize{24pt}{10pt}\selectfont\textbf{Removal of EEG Ocular Artifacts}} % Article title

\author{
\large
\textsc{John Smith}\thanks{A thank you or further information}\\[2mm] % Your name
\normalsize University of California \\ % Your institution
\normalsize \href{mailto:john@smith.com}{john@smith.com} % Your email address
\vspace{-5mm}
}
\date{}

%----------------------------------------------------------------------------------------

\begin{document}

\maketitle % Insert title

\thispagestyle{fancy} % All pages have headers and footers

%----------------------------------------------------------------------------------------
%	ABSTRACT
%----------------------------------------------------------------------------------------

\begin{abstract}

\noindent Here is the abstract.

\end{abstract}

%----------------------------------------------------------------------------------------
%	ARTICLE CONTENTS
%----------------------------------------------------------------------------------------

\begin{multicols}{2} % Two-column layout throughout the main article text


\section{Introduction}

\lettrine[nindent=0em,lines=3]{L} orem ipsum dolor sit amet, consectetur adipiscing elit.



\subsection{Related Work}



%------------------------------------------------

\section{Ocular Artifact De-noising Pipeline}
Here we give a short introduction to the pipeline we make.
\subsection{Ocular Artifact Detection}
Here we talk about how we detect ocular artifacts.
\subsection{Ocular Artifact Removal}
Here we talk about how we remove the detected ocular artifacts.
\subsection{Filter-bank Common Spatial Patterns}
Here we talk about FBSCP, what it is, how it works and why we use it.
\subsection{SVM Classification}
Here we talk about how support vector machines work and how we use it.


%------------------------------------------------

\section{Experimental Results}
In order to evaluate our method, we use the BCI Competition IV dataset 2a \citep{brunner2008bci}, which contains 4-class motor imagery EEG data from 9 subjects. The dataset consist of labeled training and test sets. Each subject participated in two sessions of 6 runs on different days. A run consist of 48 labeled trials, divided evenly between the 4 classes. Each trial measured the brain signals of a subject on 22 EEG channels and 3 EOG channels. We discarded the EOG channels since we are interested in correcting artifacts without any reference signals.

We set up two pipelines to be compared. The first consists of OACL, followed by FBCSP and Random Forest. The second is without the OACL step, i.e., just FBCSP and Random Forest. We perform 6-fold cross-validation on the training data using five runs for training and one for validation. We run 200 iterations of Bayesian Optimization of select hyperparameters for each pipeline. The best hyperparameters for each pipeline is then used for the final evaluation. We repeat this for each subject. \Cref{fig:results} shows the obtained accuracies and KAPPA scores for each subject.

\begin{table}[H]
	\centering
	\caption{Accuracy and KAPPA score for each subject.}
	\label{fig:results}
	\begin{tabular}{@{}ccc|cc@{}}
		\toprule
		\textbf{S}             & \multicolumn{2}{c|}{\textbf{W. BO OACML}} & \multicolumn{2}{c}{\textbf{W/O BO OACML}} \\ \midrule
		\multicolumn{1}{c|}{}  & Acc                   & KAPPA             & Acc                   & KAPPA             \\ \midrule
		\multicolumn{1}{c|}{1} &                       &                   &                       &                   \\
		\multicolumn{1}{c|}{2} &                       &                   &                       &                   \\
		\multicolumn{1}{c|}{3} &                       &                   &                       &                   \\
		\multicolumn{1}{c|}{4} &                       &                   &                       &                   \\
		\multicolumn{1}{c|}{5} & \textbf{}             &                   & \textbf{}             &                   \\
		\multicolumn{1}{c|}{6} &                       &                   &                       &                   \\
		\multicolumn{1}{c|}{7} &                       &                   &                       &                   \\
		\multicolumn{1}{c|}{8} &                       &                   &                       &                   \\
		\multicolumn{1}{c|}{9} &                       &                   &                       &                   \\ \bottomrule
	\end{tabular}
\end{table}

To determine the significance of these results we use the Wilcoxon signed-rank test. \Cref{fig:wilcoxon} shows the results of the test.

\begin{table}[H]
	\centering
	\caption{Wilcoxon signed-rank test}
	\label{fig:wilcoxon}
	\begin{tabular}{@{}l|llll@{}}
		\toprule
		S & No OACL & OACL & Diff & Rank \\ \midrule
		&               &                 &      &      \\
		&               &                 &      &      \\
		&               &                 &      &      \\
		&               &                 &      &      \\
		&               &                 &      &      \\
		&               &                 &      &      \\
		&               &                 &      &      \\
		&               &                 &      &      \\
		&               &                 &      &      \\ \bottomrule
	\end{tabular}
\end{table}

\subsection{Discussion}
Here we discuss the results given in section 3, and talk more about what the results imply/how it could be improved.
Remember to discuss the issues of complexity of experiment vs. how much effort we put into selecting the "best" candidate in Bayesian Optimization

%------------------------------------------------

\section{Conclusion}
\todo{to be done}

%----------------------------------------------------------------------------------------
%	REFERENCE LIST
%----------------------------------------------------------------------------------------

\begin{thebibliography}{99} % Bibliography - this is intentionally simple in this template
 
@article{blankertz2008optimizing,
	title={Optimizing spatial filters for robust EEG single-trial analysis},
	author={Blankertz, Benjamin and Tomioka, Ryota and Lemm, Steven and Kawanabe, Motoaki and Muller, Klaus-Robert},
	journal={Signal Processing Magazine, IEEE},
	volume={25},
	number={1},
	pages={41--56},
	year={2008},
	publisher={IEEE}
}
\end{thebibliography}

%----------------------------------------------------------------------------------------

\end{multicols}

\end{document}
